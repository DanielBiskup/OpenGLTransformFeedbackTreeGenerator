%%%%%%%%%%%%%%%%%%% vorlage.tex %%%%%%%%%%%%%%%%%%%%%%%%%%%%%
%
% LaTeX-Vorlage zur Erstellung von Projekt-Dokumentationen
% im Fachbereich Informatik der Hochschule Trier
%
% Basis: Vorlage svmono des Springer Verlags
%
%%%%%%%%%%%%%%%%%%%%%%%%%%%%%%%%%%%%%%%%%%%%%%%%%%%%%%%%%%%%%

\documentclass[envcountsame,envcountchap, deutsch]{i-studis}

\usepackage{makeidx}         	% Index
\usepackage{multicol}        	% Zweispaltiger Index
%\usepackage[bottom]{footmisc}	% Erzeugung von Fu�noten

%%-----------------------------------------------------
%\newif\ifpdf
%\ifx\pdfoutput\undefined
%\pdffalse
%\else
%\pdfoutput=1
%\pdftrue
%\fi
%%--------------------------------------------------------
%\ifpdf
\usepackage[pdftex]{graphicx}
\usepackage[pdftex,plainpages=false]{hyperref}
%\else
%\usepackage{graphicx}
%\usepackage[plainpages=false]{hyperref}
%\fi

%%-----------------------------------------------------
\usepackage{color}				% Farbverwaltung
%\usepackage{ngerman} 			% Neue deutsche Rechtsschreibung
\usepackage[english, ngerman]{babel}
\usepackage[latin1]{inputenc} 	% Erm�glicht Umlaute-Darstellung
%\usepackage[utf8]{inputenc}  	% Erm�glicht Umlaute-Darstellung unter Linux (je nach verwendetem Format)

%-----------------------------------------------------
\usepackage{listings} 			% Code-Darstellung
\lstset
{
	basicstyle=\scriptsize, 	% print whole listing small
	keywordstyle=\color{blue}\bfseries,
								% underlined bold black keywords
	identifierstyle=, 			% nothing happens
	commentstyle=\color{red}, 	% white comments
	stringstyle=\ttfamily, 		% typewriter type for strings
	showstringspaces=false, 	% no special string spaces
	framexleftmargin=7mm, 
	tabsize=3,
	showtabs=false,
	frame=single, 
	rulesepcolor=\color{blue},
	numbers=left,
	linewidth=146mm,
	xleftmargin=8mm
}
\usepackage{textcomp} 			% Celsius-Darstellung
\usepackage{amssymb,amsfonts,amstext,amsmath}	% Mathematische Symbole
\usepackage[german, ruled, vlined]{algorithm2e}
\usepackage[a4paper]{geometry} % Andere Formatierung
\usepackage{bibgerm}
\usepackage{array}
\hyphenation{Ele-men-tar-ob-jek-te  ab-ge-tas-tet Aus-wer-tung House-holder-Matrix Le-ast-Squa-res-Al-go-ri-th-men} 		% Weitere Silbentrennung bei Bedarf angeben
\setlength{\textheight}{1.1\textheight}
\pagestyle{myheadings} 			% Erzeugt selbstdefinierte Kopfzeile
\makeindex 						% Index-Erstellung


%--------------------------------------------------------------------------
\begin{document}
%------------------------- Titelblatt -------------------------------------
\title{Entwurf einer Command Queue f�r Echtzeit-Strategiespiele}
\subtitle{Creating a command queue for real time strategy games}
%---- Die Art der Dokumentation kann hier ausgew�hlt werden---------------
%\project{Bachelor-Projektarbeit}
%\project{Bachelor-Abschlussarbeit}
%\project{Master-Projektstudium}
%\project{Master-Abschlussarbeit}
%\project{Seminar zur Vorlesung ...}
\project{Hausarbeit zur Vorlesung K�nstliche Intelligenz f�r Spiele}
%--------------------------------------------------------------------------
\supervisor{Prof. Dr. Christof Rezk-Salama} 		% Betreuer der Arbeit
\author{Daniel Biskup, 952976} 							% Autor der Arbeit
\address{Trier,} 							% Im Zusammenhang mit dem Datum wird hinter dem Ort ein Komma angegeben
\submitdate{7.9.2014} 				% Abgabedatum
%\begingroup
%  \renewcommand{\thepage}{title}
%  \mytitlepage
%  \newpage
%\endgroup
\begingroup
  \renewcommand{\thepage}{Titel}
  \mytitlepage
  \newpage
\endgroup
%--------------------------------------------------------------------------
\frontmatter 
%--------------------------------------------------------------------------
%\input{chapters/Vorwort}				% Vorwort (optional)
%\kurzfassung

%% deutsch
\paragraph*{}


In dieser Hausarbeit habe wird untersucht in wieweit sich die Command Queue f�r Echtzeit-Strategiespiele im Zusammenhang mit als Statemachines modelierten Handlungsabl�ufen verwenden l�sst.

%--------------------------------------
In der Kurzfassung soll in kurzer und pr�gnanter Weise der wesentliche Inhalt der Arbeit beschrieben werden. Dazu z�hlen vor allem eine kurze Aufgabenbeschreibung, der L�sungsansatz sowie die wesentlichen Ergebnisse der Arbeit. Ein h�ufiger Fehler f�r die Kurzfassung ist, dass lediglich die Aufgabenbeschreibung (d.h. das Problem) in Kurzform vorgelegt wird. Die Kurzfassung soll aber die gesamte Arbeit widerspiegeln. Deshalb sind vor allem die erzielten Ergebnisse darzustellen. Die Kurzfassung soll etwa eine halbe bis ganze DIN-A4-Seite umfassen.

Hinweis: Schreiben Sie die Kurzfassung am Ende der Arbeit, denn eventuell ist Ihnen beim Schreiben erst vollends klar geworden, was das Wesentliche der Arbeit ist bzw. welche Schwerpunkte Sie bei der Arbeit gesetzt haben. Andernfalls laufen Sie Gefahr, dass die Kurzfassung nicht zum Rest der Arbeit passt.

%% englisch
\paragraph*{}
The same in english.
 			% Kurzfassung Deutsch/English
\tableofcontents 						% Inhaltsverzeichnis
%\listoffigures 							% Abbildungsverzeichnis (optional)
%\listoftables 							% Tabellenverzeichnis (optional)
%--------------------------------------------------------------------------
\mainmatter                        		% Hauptteil (ab hier arab. Seitenzahlen)
%--------------------------------------------------------------------------
% Die Kapitel werden in separaten .tex-Dateien abgelegt und hier eingebunden.

\chapter{Einleitende Hinweise} \label{Einleitende Hinweise}

Eine Anmerkung zu Beginn: Das Programm kompiliert und linkt zwar unter Windows, st�rzt beim Ausf�hren allerdings ab. Damit sie sich vorstellen k�nnen wie das Programm aussehen w�rde wenn sie es ausf�hren k�nnten, habe ich ihnen ein Video davon aufgenommen, wie das Programm aussieht, wenn es unter Linux ausgef�hrt wird. Die Videodatei hei�t \code{Video.mp4}.

Dieses Projekt enth�lt die beiden Archive \code{Projekt\_Windows} und \code{Projekt\_Linux}. Bitte ber�cksichtigen sie, dass die eigentliche Abgabe die sich nach M�glichkeit bewerten sollten \code{Projekt\_Linux} ist. Dieses enth�lt Teilweise neuere Codeabschnitte. Der Grund daf�r, dass ich zwei Projekte habe ist, dass ich das Linux Projekt auf einen Windows-Rechner kopiert habe um es dort an Windows an zu passen. Als ich es zur�ck auf mein Linux System kopierte waren die Zugriffsrechte auf einige Dateien wie Shared Libraries ver�ndert und daher lief auf meine Linux Build nicht mehr. Da meine Windowsversion auch auf Windows nicht richtig funktioniert, habe ich mich entschieden, die �nderungen f�r Windows nicht zur�ck in mein Hauptprojekt \code{Projekt\_Linux} zu integrieren. Das beiliegende Linuxprojekt sollte allerdings fehlerfrei funktionieren.

\chapter{Konzeption der Geometrieerzeugung}
In diesem Kapitel werden die Rechnungen und Definitionen hergeleitet, die ben�tigt werden, um die Geometrie des Baumes zu erzeugen.

\section{Attribute der Eckpunkte}
%Quellen: 
%  https://www.opengl.org/wiki/Face_Culling

In der Computergrafik setzen sich Objekte �blicherweise aus Dreiecken zusammen. Jedes Dreieck ist �ber drei Eckpunkte eindeutig definiert. H�ufig ist es wichtig zwischen der Vorder- und R�ckseite eines Dreiecks unterscheiden zu k�nnen, zum Beispiel um Face Culling durchzuf�hren. F�r ein auf eine Ebene projiziertes Dreieck im $\mathbb{R}^3$ kann diese Unterscheidung kann anhand der Drehrichtung in der die Eckpunkte um den Mittelpunkt des Dreiecks angeordnet sind getroffen werden. 
Sind die Punkte in mathematisch positiver Drehrichtung -- gegen den Uhrzeigersinn -- um den Mittelpunkt des Dreiecks angeordnet handelt es sich um ein Dreieck mit positiver Wicklungsrichtung (englisch. winding order) andernfalls um eines mit negativer Wicklungsrichtung.
Je nach Konvention wird entweder die Seite eines Dreiecks die in negativer oder in positiver Wicklungsrichtung erscheint als Vorderseite interpretiert. 
In der vorliegenden Arbeit verwendete Konvention entspricht der Standarteinstellung \code{glFrontFace( GL\_CCW );} von OpenGL, bei der die Seite mit positiver Wicklungsrichtung -- gegen den Uhrzeigersinn -- als Vorderseite interpretiert wird.
%Beispielgrafik:
% http://www.marekknows.com/images/forum/posts/post_722.jpg

Daher muss bei der Generierung neuer Geometrie darauf geachtet werden, dass die Punkte der generierten Dreiecke in der richten Reihenfolge angegeben werde.

Beim Erzeugen der in der Einleitung beschriebenen Geometrie ist es wichtig, dass f�r jedes Dreiecken $(p_0, p_1, p_2)$ auch die H�he $l$ des Pyramidenstumpfes gegeben ist, der m�glicherweise aus diesem generiert werden soll. Im Folgenden wird ein Dreieck $(p_0, p_1, p_2)$ mit H�he $l$ repr�sentiert durch ein Tupel $p=((p_0, p_1, p_2),l)$ mit $l \in \mathbb{N}$ wobei  $\mathbb{N} := \{0,1,2,3,\ldots \}$.

\section{Konstruktion der Geometrie}
Es sei gegeben, dass die Funktion zum erzeugen neuer Geometrie als Eingabe neben einem Dreieck $p=((p_0, p_1, p_2),l)$ zus�tzlich die drei �ber alle Iterationen hinweg konstanten Werte \code{scaleTriangle}, \code{pyramidFactor} und \code{scaleLength} erwartet. Gilt $l = 0$, so soll das Dreieck unver�ndert zur�ckgegeben werden. Gilt $l > 0$, so soll die in (KABB1) abgebildete Geometrie zur�ckgegeben werden. Im Folgenden werden zun�chst die Eckpunkte und anschlie�end aus diesen die Dreiecke der zu erzeugenden Geometrie konstruiert.

\subsection{Konstruktion der Eckpunkte}
Ist ein Dreieck $p=((p_0, p_1, p_2),l)$ mit $l > 0$ gegeben, so werden zur Erzeugung der neuen Geometrie die Punkte $p_0, p_1, p_2, q_1, q_2, q_3$ und $t$ ben�tigt (KABB1).
Da nach gew�hlter Konvention gilt, dass die Vorderseite eines Dreiecks eine positive Wicklungsrichtung hat, l�sst sich die Normale eines Dreiecks $d$ �ber das Kreuzprodukt wie folgt bestimmen:
%Q: https://de.wikipedia.org/wiki/Kreuzprodukt
$$n= \frac{a \times b}{| a \times b |} \text{ mit } a = p_1 - p_0 \text{ und } b = p_2 - p_0$$

%Q: https://de.wikipedia.org/wiki/Geometrischer_Schwerpunkt#Dreieck
Der Schwerpunkt $c_p$ des Dreiecks liegt bei $c=\frac{p_0+p_1+p_2}{3}$.
Das Dreieck $q=(q_0,q_1,q_2)$ soll kleiner sein, als das Dreieck $p$. Die Eckpunkte des Dreiecks $d$ , dessen Eckpunkte jeweils um den konstanten Faktor $0 < \code{scaleTriangle} \leq 1$ weiter von $c_p$ entfernt sind als die von $p$, lassen sich bestimmen mit $d_i = (p_i - c_p) \cdot \code{scaleTriangle}$ f�r $i \in \{0,1,2\}$. (Siehe KABB2)
Der Vektor vom Schwerpunkt $c_p$ zu jenem des Dreiecks $q$ ist $h = n \cdot l$.

Damit ergeben sich die Eckpunkte von $q$ zu:
$$q_i = c_p + d_i + h \text{ f�r } i \in \{0,1,2\}$$

Die H�he $h_{pyramid}$ der Pyramide $(q_0,q_1,q_2,t)$ wird in diesem Projekt festgelegt als ein Vielfaches der L�nge $l$. $h_{pyramid} = l * \code{pyramidFactor}$, wobei es sich bei $\code{pyramidFactor}$ um eine Konstante handelt.
Die Spitze $t$ der Pyramide liegt damit bei $t = c_p + h + h_{pyramid} * n$.

\subsection{Konstruktion der Geometrie aus den Eckpunkten}
Wenn die Punkte $p_0, p_1, p_2, q_1, q_2, q_3$ und $t$ bekannt sind, so kann aus diesen sowohl die Geometrie der Mantelfl�che, als auch die der Pyramide erzeugt werden.

\paragraph{Die Dreiecke der Mantelfl�che}
Die Dreiecke welche die Mantelfl�che des Pyramidenstumpfes bilden sollen einen L�ngenwert von $0$ haben, damit sie bei der n�chsten Iteration nicht durch weitere Geometrie ersetzt werden. Zudem sind die Eckpunkte der Dreiecke in negativer Wicklungsrichtung an zu geben, damit es auch im n�chsten Schritt m�glich ist die Normalen der Dreiecke zu berechnen und damit OpenGL Backface Culling durchf�hren kann. Wie Abbildung KAAB3 entnommen werden kann, ergibt sich die Menge der Dreiecke der Mantelfl�che zu:
$$M := \{(p_0, q_1, q_0), (p_0,p_1,q_1), (p_1, q_2, q_1), (p_1,p_2,q_2), (p_2, q_0, q_2), (p_2,p_0,q_0)\}
$$
Oder k�rzer:
$$M := \{x \in \{(p_i, q_j, q_i), (p_i,p_j,q_j)\} | i \in \{0,1,2\} \wedge j = (i + 1) \bmod 3 \}
$$
Die Menge $M_L$ aller Dreiecke mit L�ngenangabe ist damit
$$M_L = \{(d,l) | d \in M \wedge l = 0\}$$

Die Dreiecke $(p_0, p_1, p_2)$ und $q_0,q_1,q_2$ sind nicht als Teil von $M_L$ definiert worden, da sie beim betrachten der fertigen Geometrie auf dem Bildschirm ohnehin verdeckt sein w�rden. Das die Geometrie des fertigen Baumes auf der Unterseite des Stammes ein dreieckiges Loch haben wird, ist unproblematisch, da B�ume f�r gew�hnlich in der Erde stecken.

\paragraph{Die Dreiecke der Pyramide}
Die Dreiecke welche die Pyramide bilden sollen in der n�chsten Iterationen um einen Bruchteil der L�nge $l$ extrudiert werden. Daher wird als L�nge f�r diese Dreiecke $l_{next} = l * \code{scaleLength}$ gew�hlt mit $0 \leq \code{scaleLength} \leq 1$.
Die Menge $P$ der Dreiecke der Pyramide l�sst sich beschreiben durch $P := \{(q_0,q_1,t), (q_0,q_1,t), (q_0,q_1,t) \}$ (siehe KABB4).  Die Menge $P_L$ der Dreiecke der Pyramide mit L�ngenangabe ist damit $P_L = \{(d,l_{next}) | d \in P \wedge l_{next} = l * \code{scaleLength}\}$.

Die Menge $D$ aller Dreiecke der in (KABB1) abgebildeten Geometrie ergibt sich damit zu $D = M_L \cup P_L$. $D$ hat eine M�chtigkeit von neun.

\section{Herleitung der Formel zur Berechnung der Anzahl generierter Dreiecke nach n Iterationen}
Beim absetzen eines Draw Calls an OpenGL �ber die Methode \code{void glDrawArrays( 	GLenum mode, GLint first, GLsizei count);} muss als dritter Parameter angegeben werden, wie viele Eckpunkte zu zeichnen sind. Da die in diesem Kapitel generierte Geometrie aus einzelnen Dreiecken zusammensetzt, und jedes Dreieck sich aus drei Eckpunkten besteht, ist die Anzahl der Eckpunkte in der Geometrie dreimal so gro� wie die der Dreiecke. Um die Anzahl der Eckpunkte in der Geometrie zu bestimmen ist es daher hilfreich zun�chst die Anzahl der Dreiecke aus denen diese sich zusammensetzt zu bestimmen.

Nach $n$ Iterationen besteht die Geometrie des generierten Baumes aus $4*3^n-3$ Eckpunkten. Diese Formel wird im Folgenden hergeleitet.

\paragraph{Herleitung}
F�r jedes Eingabedreieck $p=((p_0, p_1, p_2),l)$, f�r welches $l=0$ gilt wird keine neue Geometrie erzeugt, die Gesamtzahl der Dreiecke bleibt unver�ndert. Gilt $l>0$ so wird das Eingabedreieck durch neun andere Dreiecke ersetzt, die Gesamtzahl steigt um acht Dreiecke.

In jeder Iteration wird jedes Dreieck mit $l>0$ ersetzt durch drei neue Dreiecke mit $l>0$. Also enth�lt die Eingabemenge der Dreiecke nach $n$ Iterationen genau $3^{n}$ Dreiecke mit $l>0$. Da mit jeder Iteration f�r jedes Dreieck mit $l>0$ genau acht neue Dreiecke hinzukommen, und nach $m$ Iterationen $3^{m}$ Dreiecke in der Eingabemenge enthalten sind die  diese Bedingung erf�llen, kommen in Iteration $n$ genau $3^n$ Dreiecke hinzu f�r $n \geq 0$. F�r die Anzahl der Dreiecke $f$ die in Iteration $n$ hinzukommen gilt daher $f(n) := 3^n * 8$.

Die Gesamtzahl $g(n)$ der Dreiecke nach $n$ Iterationen ist damit:
%Quelle: Summenrechnung
$$g(n) := 1 + \sum_{i = 0}^{n-1} f(i)$$
$$= 1 + \sum_{i = 0}^{n-1} 8 * 3^i$$
$$= 1 + 8 * \sum_{i = 0}^{n-1} 3^i$$
$$= 1 + 8 * ((\sum_{i = 0}^{n} 3^i) - 3^n )$$
$$= 1 + 8 * (\frac{1-3^{n+1}}{1-3} - 3^n ) $$
$$= 1 + 8 * (\frac{1-3^{n+1}}{-2} - 3^n )$$
$$= 1 + 8 * (\frac{1-3^{n+1}+2*3^n}{-2})$$
$$= 1 + 8 * (\frac{1-3*3^{n}+2*3^n}{-2})$$
$$= 1 + 8 * (\frac{1-3^n}{-2})$$
$$= 1 + (\frac{4*(1-3^n)}{-1})$$
$$= 1 - (4 -4 * 3^n)$$
$$= 4 * 3^n - 3$$

Da f�r $n = 0$ gilt $g(0) = 4 * 3^0 - 3 = 1$, gilt:
$$\forall n \in \mathbb{N}:g(n) := 4 * 3^n - 3$$

\chapter{Implementierung}
Dieses Kapitel behandelt das diesem Dokument beiliegende Programm.
Zun�chst wird �berblick dar�ber gegeben, wie das Programm zu bedienen ist. Anschlie�end wird auf die verwendeten Libraries, und den Aufbau des Programmes eingegangen. Zuletzt wird ausgef�hrt wie das Programm auf den Plattformen Windows und Linux kompiliert werden kann.

\subsection{Bedienung des Programmes}
Nach dem Start des Programmes ist in der Mitte des Fensters ein sich drehender Baum zu sehen. In der linken oberen Ecke befindet sich ein Fenster mit der mit dem Titel "TweakBar" welches im folgenden als Tweak Bar bezeichnet wird.

Die Tweak Bar enth�lt drei Gruppen von Eintr�gen, mit Namen "generation parameters", "presentation parameters" und "read-only scene information".

W�hrend die ersten Beiden Gruppen Felder die vom Benutzer oder der Benutzerin manipuliert werden k�nnen enthalten, enth�lt die letzte Gruppe die nicht vom Benutzer oder der Benutzerin nicht direkt manipulierbaren Eintr�ge "number of triangles" und "number of vertices". Unter "number of triangles" ist die Anzahl der Dreiecke und unter "number of vertices" anderen die Anzahl der Eckpunkte aufgef�hrt, aus denen sich der aktuell auf dem Bildschirm angezeigt Baum zusammensetzt.

In der Gruppe "presentation parameters" findet der Benutzer oder die Benutzerin den Schalter "autoRotation" mit dem sich festlegen l�sst, ob der Baum automatisch entlang seiner Y-Achse gedreht werden soll, sowie das Eingabefeld "autoRotationSpeed" �ber welches die bei automatischer Rotation anzuwendende Rotationsgeschwindigkeit festgelegt werden kann.
�ber den Eintrag "rotation" kann durch klicken und ziehen der bunt eingef�rbten Kugel die Orientierung des Baumes im Raum ver�ndert werde.

In der Gruppe "generation parameters" sind Eintr�ge aufgef�hrt �ber die sich der Prozess der Geometriegenerierung des Baumes konfigurieren l�sst.
Den ersten Eintrag bildet ein Button mit Aufschrift "click here to generate tree". Ein Klick auf diesen l�st die Ausf�hrung des in [Kapite X] beschriebenen Geometrieerzeugungsalgorithmus aus, welchem als Parameter die Werte der �brigen Eingabefelder, die in der Gruppe "generation parameters" enthalten sind, �bergeben werde. Wurde die Geometrie erfolgreich erzeugt, so wird der diese nun anstelle der zuvor angezeigten angezeigt.

Der Parameter "numberOfIterations" gibt an, wie viele Iterationen des Geometrieerzeugungsalgorithmus verwendet werden sollen.

Die �brigen Parameter sind bereits aus [Kapitel X] bekannt, ihr Bedeutung soll hier dennoch kurz gekl�rt werden.

Der Parameter "scaleLength" bestimmt, wie lang ein Astsegment im Verh�ltnis zu seinem Vorg�nger ist.

Der Parameter "scaleTriangle" bestimmt, wie dick das obere Ende eines Astsegmentes im Verh�ltnis zu seinem Unteren Ende ist.

%TODO: Stimmt diese Aussage noch?
Am oberen Ende eines Astsegmentes befindet sich eine Pyramide. Die Pyramide hat als Grundfl�che jenes Dreieck dessen Gr��e durch "scaleTriangle" bestimmt wird. Wie hoch diese Pyramide Verh�ltnis zu der Seitenl�nge des Dreieckes welches ihre Basis bildet ist, bestimmt das Attribut "pyramidFactor".

%Die �u�ere Gestaltung der Ausarbeitung hinsichtlich %Abschnittformate, Abbildungen, mathematische Formeln %usw. wird in \hyperref[Stile]{Kapitel~\ref*{Stile}} %kurz dargestellt.

\subsection{Verwendet Libraries}
Das Programm nutzt mehrere Programmbibliotheken.


\paragr
\subsection{Aufbau des Programmes}

\subsection{Der Build Prozess}



%\chapter{Zusammenfassung und Ausblick}

\section{Fazit}
Das Beispielprogramm bietet die M�glichkeit Einheiten eine Reihe von Befehlen zu erteilen welche sie nacheinander ausf�hren sollen.
Bisher sind zwar nur die zwei Command-Klassen \verb+CollectWoodCommad+ und \verb+GoToPositionCommand+, doch weitere Commands sowie korrespondierende StateMachine Kompositionen, k�nnen ohne Probleme in die Anwendung eingegliedert werden. Allerdings stellt sich die Kompositionen komplexerer StateMachines aus den grundlegenden States und StateMachines als recht un�bersichtlich und fehleranf�llig heraus.
Grunds�tzlich jedoch, scheint die in dieser Hausarbeit gezeigt Herangehensweise Command Queues zu implementieren zumindest f�r kleinere Projekte vollkommen au�reichend zu sein.

\section{Ausblick}
Auf Basis der Beispielanwendung k�nnte man durchaus noch aufbauen. Die m�glichen Erweiterungen sind so mannigfaltig, dass sie leider nicht im Rahmen dieser Hausarbeit realisiert werden konnen.

\subsection{Hinzuf�gen und ab�ndern von Charactertypen, Verhalten und Geb�ude}
Denkbar w�re die Implementierung von Geb�udetypen wie Kaserne welche �ber die Command Queue Befehle zum Bau von Einheiten entgegennehmen k�nnen.

Es k�nnte ein Patrouilleverhalten implementiert werden, welches es erlaubt einem Character den Befehl zu erteilen immer die selbe Route ab zu laufen.

B�ume k�nnten so abge�ndert werden, dass sie nur noch eine begrenzte Menge Holz abgeben. Dazu w�rde man eine TreeComponent entwerfen und implementieren.

Die m�glichkeit Geb�ude zu errichten k�nnte implementiert werden.

\subsection*{Kollisionsvermeidung, Wegfindung, Formationen}
Damit die Simulation glaubw�rdiger erscheint, w�re es hilfreich daf�r zu sorgen, dass Einheiten nicht mehr miteinander kollidieren k�nnen. Setzt man dies um, so muss man allerdings auch einen Wegfindungsalgorithmus wie A* implementieren welcher daf�r sorgt, dass Einheiten einen Weg zu ihrem Ziel finden obwohl der direkte Weg blockiert ist. Denkt man diesen Gedanken weiter ergibt sich daraus die Idee, dass eigene Einheiten die einer Einheit des selben Teams im Weg stehen, dies mitgeteilt bekommen und versuchen die Blockade aufzul�sen.

Sobald man mehre Einheiten hat, k�nnte es sich auch anbieten �ber Gruppenformationen (wie sie in ''Artificial Intelligence for Games'' von Ian Millington in Kapitel 3.7.3 beschrieben) nachzudenken.

\subsection{High-Level und Intermediate-Level AI}
H�here Ebenen der AI, welche die bereits vorhandene Command Queue-Infrastruktur verwenden w�rden, k�nnten entwickelt werden. Zum Beispiel AIs welche sich um den Basisbau k�mmern.

Ohne hier weiter ins Detail zu gehen m�chte ich auf das Kapitel 8.2 in ''AI Game Programming Wisdom'' verweisen in welchem die �blichen Schichten einer AI f�r Echtzeit-Strategiespiele aufgef�hrt sind.

\subsection{Editor f�r die Komposition von StateMachines}
Es w�re denkebar einen grafischen Editor f�r zu entwickeln, der die Komposition von StateMachines aus bereis vorhandenen States und StateMachines aus \verb+stats/basicStates+ erleichtert. Als m�gliches Ausgabevormat w�rde f�r den Anfang schon C++ Quelltext ausreichen. Ein solcher Editor k�nnt zum Beispiel im Rahmen einer anderen Vorlesung umgesetzt werden.




% ...
%--------------------------------------------------------------------------
%\backmatter                        		% Anhang
%-------------------------------------------------------------------------
%\bibliographystyle{geralpha}			% Literaturverzeichnis
%\bibliography{literatur}     			% BibTeX-File literatur.bib
%--------------------------------------------------------------------------
%printindex 							% Index (optional)
%--------------------------------------------------------------------------
%\begin{appendix}	
   %\include{chapters/Implementierungsdetails}					
   %\chapter{Zusammenfassung und Ausblick}




\section{Erweiterbarkeit}
Das Beispielprogramm kennt zwar nur die zwei Command-Klassen \verb+CollectWoodCommad+ und \verb+GoToPositionCommand+ doch weitere Commands sowie StateMachine Kompositionen, welche die Ausf�hrung dieser repr�sentieren, k�nnen nach dem Vorgestellten Vorgehen ohne Probleme in die Anwendung eingef�gt werden.


\section{Ausblick}
-- H�here Ebene der AI

%---ANHANG

\section{Hinweise f�r Entwickler und Hinweise auf Implementierungsdetails}
\subsection{Components als Voraussetzung f�r States und StatMachines}
Es bietet sich an dieses Vorhandensein ben�tigter Components schon im Konstruktor des States mit assert() zu �berpr�fen. Sind diese nicht Vorhanden wird es fr�her oder sp�ter ohnehin zum Absturz des Systems kommen.

\subsection{StateMachines als State verwenden}
Die StateMachine ist im Gegensatz zur L�sung der �bungsaufgabe so implementiert, dass eine 

 			% z.B. Grafiken von StateMachine Kompositionen und so. 
   %\include{chapters/Glossar}			% Glossar   
   %\include{chapters/Selbststaendigkeitserklaerung}	% Selbstst�ndigkeitserkl�rung
%\end{appendix}

\end{document}
