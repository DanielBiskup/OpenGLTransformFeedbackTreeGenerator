
\chapter{Einleitende Hinweise} \label{Einleitende Hinweise}

Eine Anmerkung zu Beginn: Das Programm kompiliert und linkt zwar unter Windows, st�rzt beim Ausf�hren allerdings ab. Damit sie sich vorstellen k�nnen wie das Programm aussehen w�rde wenn sie es ausf�hren k�nnten, habe ich ihnen ein Video davon aufgenommen, wie das Programm aussieht, wenn es unter Linux ausgef�hrt wird. Die Videodatei hei�t \code{Video.mp4}.

Dieses Projekt enth�lt die beiden Archive \code{Projekt\_Windows} und \code{Projekt\_Linux}. Bitte ber�cksichtigen sie, dass die eigentliche Abgabe die sich nach M�glichkeit bewerten sollten \code{Projekt\_Linux} ist. Dieses enth�lt Teilweise neuere Codeabschnitte. Der Grund daf�r, dass ich zwei Projekte habe ist, dass ich das Linux Projekt auf einen Windows-Rechner kopiert habe um es dort an Windows an zu passen. Als ich es zur�ck auf mein Linux System kopierte waren die Zugriffsrechte auf einige Dateien wie Shared Libraries ver�ndert und daher lief auf meine Linux Build nicht mehr. Da meine Windowsversion auch auf Windows nicht richtig funktioniert, habe ich mich entschieden, die �nderungen f�r Windows nicht zur�ck in mein Hauptprojekt \code{Projekt\_Linux} zu integrieren. Das beiliegende Linuxprojekt sollte allerdings fehlerfrei funktionieren.
