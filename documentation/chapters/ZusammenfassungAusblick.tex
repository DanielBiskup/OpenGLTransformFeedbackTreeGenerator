\chapter{Zusammenfassung und Ausblick}

\section{Fazit}
Das Beispielprogramm bietet die M�glichkeit Einheiten eine Reihe von Befehlen zu erteilen welche sie nacheinander ausf�hren sollen.
Bisher sind zwar nur die zwei Command-Klassen \verb+CollectWoodCommad+ und \verb+GoToPositionCommand+, doch weitere Commands sowie korrespondierende StateMachine Kompositionen, k�nnen ohne Probleme in die Anwendung eingegliedert werden. Allerdings stellt sich die Kompositionen komplexerer StateMachines aus den grundlegenden States und StateMachines als recht un�bersichtlich und fehleranf�llig heraus.
Grunds�tzlich jedoch, scheint die in dieser Hausarbeit gezeigt Herangehensweise Command Queues zu implementieren zumindest f�r kleinere Projekte vollkommen au�reichend zu sein.

\section{Ausblick}
Auf Basis der Beispielanwendung k�nnte man durchaus noch aufbauen. Die m�glichen Erweiterungen sind so mannigfaltig, dass sie leider nicht im Rahmen dieser Hausarbeit realisiert werden konnen.

\subsection{Hinzuf�gen und ab�ndern von Charactertypen, Verhalten und Geb�ude}
Denkbar w�re die Implementierung von Geb�udetypen wie Kaserne welche �ber die Command Queue Befehle zum Bau von Einheiten entgegennehmen k�nnen.

Es k�nnte ein Patrouilleverhalten implementiert werden, welches es erlaubt einem Character den Befehl zu erteilen immer die selbe Route ab zu laufen.

B�ume k�nnten so abge�ndert werden, dass sie nur noch eine begrenzte Menge Holz abgeben. Dazu w�rde man eine TreeComponent entwerfen und implementieren.

Die m�glichkeit Geb�ude zu errichten k�nnte implementiert werden.

\subsection*{Kollisionsvermeidung, Wegfindung, Formationen}
Damit die Simulation glaubw�rdiger erscheint, w�re es hilfreich daf�r zu sorgen, dass Einheiten nicht mehr miteinander kollidieren k�nnen. Setzt man dies um, so muss man allerdings auch einen Wegfindungsalgorithmus wie A* implementieren welcher daf�r sorgt, dass Einheiten einen Weg zu ihrem Ziel finden obwohl der direkte Weg blockiert ist. Denkt man diesen Gedanken weiter ergibt sich daraus die Idee, dass eigene Einheiten die einer Einheit des selben Teams im Weg stehen, dies mitgeteilt bekommen und versuchen die Blockade aufzul�sen.

Sobald man mehre Einheiten hat, k�nnte es sich auch anbieten �ber Gruppenformationen (wie sie in ''Artificial Intelligence for Games'' von Ian Millington in Kapitel 3.7.3 beschrieben) nachzudenken.

\subsection{High-Level und Intermediate-Level AI}
H�here Ebenen der AI, welche die bereits vorhandene Command Queue-Infrastruktur verwenden w�rden, k�nnten entwickelt werden. Zum Beispiel AIs welche sich um den Basisbau k�mmern.

Ohne hier weiter ins Detail zu gehen m�chte ich auf das Kapitel 8.2 in ''AI Game Programming Wisdom'' verweisen in welchem die �blichen Schichten einer AI f�r Echtzeit-Strategiespiele aufgef�hrt sind.

\subsection{Editor f�r die Komposition von StateMachines}
Es w�re denkebar einen grafischen Editor f�r zu entwickeln, der die Komposition von StateMachines aus bereis vorhandenen States und StateMachines aus \verb+stats/basicStates+ erleichtert. Als m�gliches Ausgabevormat w�rde f�r den Anfang schon C++ Quelltext ausreichen. Ein solcher Editor k�nnt zum Beispiel im Rahmen einer anderen Vorlesung umgesetzt werden.


