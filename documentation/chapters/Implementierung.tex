\chapter{Implementierung}
Dieses Kapitel behandelt das diesem Dokument beiliegende Programm.
Zun�chst wird �berblick dar�ber gegeben, wie das Programm zu bedienen ist. Anschlie�end wird auf die verwendeten Libraries, und den Aufbau des Programmes eingegangen. Zuletzt wird ausgef�hrt wie das Programm auf den Plattformen Windows und Linux kompiliert werden kann.

\subsection{Bedienung des Programmes}
Nach dem Start des Programmes ist in der Mitte des Fensters ein sich drehender Baum zu sehen. In der linken oberen Ecke befindet sich ein Fenster mit der mit dem Titel "TweakBar" welches im folgenden als Tweak Bar bezeichnet wird.

Die Tweak Bar enth�lt drei Gruppen von Eintr�gen, mit Namen "generation parameters", "presentation parameters" und "read-only scene information".

W�hrend die ersten Beiden Gruppen Felder die vom Benutzer oder der Benutzerin manipuliert werden k�nnen enthalten, enth�lt die letzte Gruppe die nicht vom Benutzer oder der Benutzerin nicht direkt manipulierbaren Eintr�ge "number of triangles" und "number of vertices". Unter "number of triangles" ist die Anzahl der Dreiecke und unter "number of vertices" anderen die Anzahl der Eckpunkte aufgef�hrt, aus denen sich der aktuell auf dem Bildschirm angezeigt Baum zusammensetzt.

In der Gruppe "presentation parameters" findet der Benutzer oder die Benutzerin den Schalter "autoRotation" mit dem sich festlegen l�sst, ob der Baum automatisch entlang seiner Y-Achse gedreht werden soll, sowie das Eingabefeld "autoRotationSpeed" �ber welches die bei automatischer Rotation anzuwendende Rotationsgeschwindigkeit festgelegt werden kann.
�ber den Eintrag "rotation" kann durch klicken und ziehen der bunt eingef�rbten Kugel die Orientierung des Baumes im Raum ver�ndert werde.

In der Gruppe "generation parameters" sind Eintr�ge aufgef�hrt �ber die sich der Prozess der Geometriegenerierung des Baumes konfigurieren l�sst.
Den ersten Eintrag bildet ein Button mit Aufschrift "click here to generate tree". Ein Klick auf diesen l�st die Ausf�hrung des in [Kapite X] beschriebenen Geometrieerzeugungsalgorithmus aus, welchem als Parameter die Werte der �brigen Eingabefelder, die in der Gruppe "generation parameters" enthalten sind, �bergeben werde. Wurde die Geometrie erfolgreich erzeugt, so wird der diese nun anstelle der zuvor angezeigten angezeigt.

Der Parameter "numberOfIterations" gibt an, wie viele Iterationen des Geometrieerzeugungsalgorithmus verwendet werden sollen.

Die �brigen Parameter sind bereits aus [Kapitel X] bekannt, ihr Bedeutung soll hier dennoch kurz gekl�rt werden.

Der Parameter "scaleLength" bestimmt, wie lang ein Astsegment im Verh�ltnis zu seinem Vorg�nger ist.

Der Parameter "scaleTriangle" bestimmt, wie dick das obere Ende eines Astsegmentes im Verh�ltnis zu seinem Unteren Ende ist.

%TODO: Stimmt diese Aussage noch?
Am oberen Ende eines Astsegmentes befindet sich eine Pyramide. Die Pyramide hat als Grundfl�che jenes Dreieck dessen Gr��e durch "scaleTriangle" bestimmt wird. Wie hoch diese Pyramide Verh�ltnis zu der Seitenl�nge des Dreieckes welches ihre Basis bildet ist, bestimmt das Attribut "pyramidFactor".

%Die �u�ere Gestaltung der Ausarbeitung hinsichtlich %Abschnittformate, Abbildungen, mathematische Formeln %usw. wird in \hyperref[Stile]{Kapitel~\ref*{Stile}} %kurz dargestellt.

\subsection{Verwendet Libraries}
Das Programm nutzt mehrere Programmbibliotheken.


\paragr
\subsection{Aufbau des Programmes}

\subsection{Der Build Prozess}

